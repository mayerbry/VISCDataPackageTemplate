\section{Background}
A comparative repeated low-dose (RLD) challenge study was conducted in rhesus macaques to explore an HIV vaccination strategy that induces affinity-matured HIV Env antibodies in the absence of potentially detrimental HIV-specific T helper cells. The vaccine groups were immunized on week 0 and 4 with adenoviral vectors encoding FIV Gag (Group A), SIV GagPol (Group C), HIV Env (Group D), or a control antigen (RSV-F, group B), with one group remaining unvaccinated (Mock, group E), as outlined in Table~\ref{schema}. At weeks 12, 20 and 28, vaccinated monkeys were either boosted by virus like particles (VLPs) containing FIV Gag and HIV Env (FIV--Henv) or SIV Gag and HIV Env (SIV--Henv). Efficacy was determined by an intrarectal repeated low dose challenge with SHIV at week 38 (10 weeks following the final boost). Sequential challenges (SHIV SF162 P3 ($3 \times 10^{4}$  $\textrm{TCID}_{50}$)) were conducted at weekly intervals with the last challenge performed at week 49.

\begingroup\small
\begin{table}[h]
\centering
\begin{tabular}{ccp{2.8cm}p{2.8cm}p{2.2cm}p{2.2cm}p{2.2cm}}
\hline
Group & N = 32 & Week 0 & Week 4 & Week 12 & Week 20 & Week 28 \\
\hline
A & 8 & Ad-FIV Gag\footnotemark[1] & Ad-FIV Gag\footnotemark[1] & FIV--Henv\footnotemark[2] & FIV--Henv\footnotemark[2] & FIV--Henv\footnotemark[2] \\
B & 6 & Ad-RSF-F\footnotemark[1] & Ad-RSF-F\footnotemark[1] & FIV--Henv\footnotemark[2] & FIV--Henv\footnotemark[2] & FIV--Henv\footnotemark[2] \\
C & 6 & Ad-SIV GagPol\footnotemark[1] & Ad-SIV GagPol\footnotemark[1] &  SIV--Henv\footnotemark[2] &  SIV--Henv\footnotemark[2] &  SIV--Henv\footnotemark[2] \\
D & 6 & Ad-Henv\footnotemark[1] & Ad-Henv\footnotemark[1] &  SIV--Henv\footnotemark[2] &  SIV--Henv\footnotemark[2] &  SIV--Henv\footnotemark[2] \\
E & 6 & Mock & Mock & Mock & Mock & Mock \\
\hline
\end{tabular}
\caption{Study Schema}
\label{schema}
\end{table}
\footnotetext[1]{$1\times 10^{10}$ particles}
\footnotetext[2]{30 $\mu$g gp120}
\endgroup
